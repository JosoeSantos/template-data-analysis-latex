
\documentclass[12pt,a4paper]{article}

% Pacotes básicos
\usepackage[utf8]{inputenc}
\usepackage[T1]{fontenc}
\usepackage[brazil]{babel}
\usepackage{graphicx}
\usepackage{float}
\usepackage{amsmath, amssymb}
\usepackage{hyperref}
\usepackage{caption}
\usepackage{cite}
\usepackage{listings} % para formatar blocos de código
\usepackage{enumitem} % para controlar listas
\usepackage{xcolor}   % necessário para cores no listings

% Configurações do listings
\lstset{
  language=Python,
  basicstyle=\ttfamily\small,
  numbers=left,
  numberstyle=\tiny,
  frame=single,
  breaklines=true,
  keywordstyle=\color{blue}\bfseries,
  stringstyle=\color{red},
  commentstyle=\color{green!60!black}\itshape,
  showstringspaces=false
}

\graphicspath{{./}{../}{../../}{../../plots/}{../../plots/img/}}

% \title{Relatório do Projeto de Grafos e Algoritmos de Busca}
% \author{
% Áquila Oliveira Souza | 2021019327 \\ 
% Felippe Veloso Marinho | 2021072260 \\ 
% Jefferson Pereira de Souza | 2022099049
% }
% \date{\today}

\begin{document}

% ==============================
% CAPA
% ==============================
\begin{titlepage}
    \centering
    {\Large \textbf{Universidade Federal de Minas Gerais}}\\[0.3cm]
    {\large Engenharia de Sistemas}\\[2cm]
    
    {\Huge \textbf{Relatório do Projeto de Grafos e Algoritmos de Busca}}\\[1.5cm]
    
    \textbf{Fundamentos de Inteligência Artificial}\\[0.5cm]
    \textbf{Professores:} Cristiano Castro e João Pedro Campos\\[1.5cm]
    
    \begin{flushleft}
        \textbf{Alunos:}\\
        Áquila Oliveira Souza --- 2021019327\\
        Arthur Jorge --- 2022055718\\
        Felippe Veloso Marinho --- 2021072260\\
        Jefferson Pereira de Souza --- 2022099049\\
        Josoé Santos Queiroz --- 2019026982
    \end{flushleft}
    
    \vfill
    {\large Belo Horizonte, MG}\\
    {\large \today}
\end{titlepage}

\clearpage
\tableofcontents
\clearpage

% ==============================
% INTRODUÇÃO
% ==============================
\section{Introdução}

\section{Modelagem do Problema em Forma de Grafo}

O problema da \textit{Ponte e da Tocha} foi modelado como um problema de busca em espaço de estados e também como um CSP (\textit{Constraint Satisfaction Problem}). Essa abordagem permite tanto a análise formal do problema em termos de grafos, como também a aplicação de algoritmos de busca clássicos para encontrar soluções ótimas. 

\subsection{Estados}
\subsection{Definição Formal do Grafo}

\subsection{Algoritmo de Dijkstra}
\section{Referências}
\clearpage
\addcontentsline{toc}{section}{Referências}
\bibliographystyle{plain}
\bibliography{references}

\end{document}
